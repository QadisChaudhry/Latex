\documentclass[12pt]{article}
\usepackage{indentfirst}
\usepackage{amsmath}
\usepackage{multicol}
\setlength{\jot}{2ex}
\usepackage{mathrsfs}
\usepackage{graphicx}
\usepackage{wrapfig}
\usepackage{booktabs}
\usepackage[letterpaper, margin=1in]{geometry}
\usepackage{fancyhdr}
\usepackage [autostyle, english = american]{csquotes}
\MakeOuterQuote{"}
\renewcommand{\baselinestretch}{1.0}
\newcommand{\objects}[2]{%
  \leavevmode\vbox{\hbox{#1}\nointerlineskip\hbox{#2}}%
}
\title{Lab 1 \\ Introduction and Number Representation}
\author{Qadis Chaudhry}
\date{September 29, 2021}
\begin{document}
\maketitle
    \section{Unsigned Integers}
    \subsection*{Conversions}
    \begin{itemize}
        \item[a.] Convert the following from their initial radix to the other
            two common radices:
    \end{itemize}
    0b011000101:
    \par $2^{7} + 2^{6} + 2^{2} + 2^{0} = \boxed{(198)_{10}}$
    \par $(1100)_{2}(0101)_{2} = (C)_{16}(5)_{16} = \boxed{(C5)_{16}}$ \\
    0xF54E:
    \par $15 \times 16^{3} + 5 \times 16^{2} + 4 \times 16 + 14 = \boxed{(62798)_{10}}$
    \par $(F)_{16}(5)_{16}(4)_{16}(E)_{16} =
    (1111)_{2}(0101)_{2}(0100)_{2}(1110)_{2} = \boxed{(1111010101001110)_{2}}$ \\
    312:
    \\
    \begin{minipage}{0.5\linewidth}
        \begin{align*}
            312 \div 2 = 156\ R&0 \\
            156 \div 2 = 78\ R&0 \\
            78 \div 2 = 39\ R&1 \\
            39 \div 2 = 19\ R&1 \\
            19 \div 2 = 9\ R&1 \\
            9 \div 2 = 4\ R&1 \\
            4 \div 2 = 2\ R&0 \\
            2 \div 2 = 1\ R&0 \\
            1 \div 2 = 0\ R&1 \\
            \boxed{(100111100)_{2}}
        \end{align*}
    \end{minipage}
    \begin{minipage}{0.5\linewidth}
        \begin{align*}
            312 \div 16 = 19\ R&8 \\
            19 \div 16 = 1\ R&3 \\
            1 \div 16 = 0\ R&1 \\
            \boxed{(138)_{16}}
        \end{align*}
    \end{minipage}
    \newpage
    \noindent 0b10110111:
    \par $2^{7} + 2^{5} + 2^{4} + 2^{2} + 2^{1} + 2^{0} = \boxed{(183)_{10}}$
    \par $(1011)_{2}(0111)_{2} = \boxed{(B7)_{16}}$ \\
    0x6BCC:
    \par $6 \times 16^{3} + 11 \times 16^{2} + 12 \times 16 + 12 = \boxed{(27596)_{10}}$
    \par $(6)_{16}(B)_{16}(C)_{16}(C)_{16} =
    (0110)_{2}(1011)_{2}(1100)_{2}(1100)_{2} = \boxed{(110101111001100)_{2}}$ \\
    87:
    \\
    \begin{minipage}{0.5\linewidth}
        \begin{align*}
            87 \div 2 = 43\ R&1 \\
            43 \div 2 = 21\ R&1 \\
            21 \div 2 = 10\ R&1 \\
            10 \div 2 = 5\ R&0 \\
            5 \div 2 = 2\ R&1 \\
            2 \div 2 = 1\ R&0 \\
            1 \div 2 = 0\ R&1 \\
            \boxed{(1010111)_{2}}
        \end{align*}
    \end{minipage}
    \begin{minipage}{0.5\linewidth}
        \begin{align*}
            87 \div 16 = 5\ R&7 \\
            5 \div 16 = 0\ R&5 \\
            \boxed{(57)_{16}}
        \end{align*}
    \end{minipage}
    \begin{itemize}
        \item[b.] Write the following using IEC prefixes:
    \end{itemize}
    \par $2^{29}: \boxed{\text{512 Mi}},\ 2^{54}: \boxed{\text{16 Pi}},\
    2^{16}: \boxed{\text{64
    Ki}},\ 2^{65}: \boxed{\text{32 Ei}},\ 2^{33}: \boxed{\text{8 Gi}},\
    2^{42}: \boxed{\text{4 Ti}}$
    \begin{itemize}
        \item[c.] Write the following using SI prefixes:
    \end{itemize}
    \par $10^{5}: \boxed{\text{100 K}}, 10^{16}: \boxed{\text{10 P}}, 10^{10}:
    \boxed{\text{10 G}}, 10^{19}: \boxed{\text{10 E}}, 10^{27}:
    \boxed{\text{1000 Y}}, 10^{4}: \boxed{\text{10 K}}$
    \begin{itemize}
        \item[d.] Write the following with powers of 10:
    \end{itemize}
    \par $\text{21 Z}: \boxed{21 \times 10^{21}}, \text{9 Y}: \boxed{9 \times 10^{24}},
    \text{13 K}: \boxed{13 \times 10^{3}}$
    \begin{itemize}
        \item[e.] Write the following with powers of 2:
    \end{itemize}
    \par $\text{14 Mi}: \boxed{14 \times 2^{20}}, \text{12 Ei}: \boxed{12 \times
    2^{60}}, \text{27 Ki}: \boxed{27 \times 10^{10}}$
    \section{Signed Integers}
    \subsection*{Two's Complement Exercises}
    \begin{itemize}
        \item[1.] What is the largest integer that can be represented with 16
            bits? How many bits do you need to represent the largest integer
            plus 1?
    \end{itemize}
    \par With Two's Complement, the range is from $-2^{16-1}$ to $2^{16-1}
    - 1$, which is from -32,768 to 32,767. This makes the largest number able to
    be represented by Two's Complement, 32,767. To represent the largest integer
    plus 1, you would need one more bit as this adds to the overall range making
    the new range from, -65536 to 65535.
    \par With unsigned numbers, the largest number that can be represented is
    $2^{16} - 1 = 65535$. To represent the largest number plus one, once again,
    you would need 17 bits.
    \begin{itemize}
        \item[2.] How do you represent the numbers 100, 6, and -5 (assume the
            numbers are 8 bits)?
    \end{itemize}
    100:
    \par For positive integers, unsigned and Two's Complement are exactly the
    same, so for 100, using the division technique,
    \begin{align*}
        100 \div 2 = 50\ R&0 \\
        50 \div 2 = 25\ R&0 \\
        25 \div 2 = 12\ R&1 \\
        12 \div 2 = 6\ R&0 \\
        6 \div 2 = 3\ R&0 \\
        3 \div 2 = 1\ R&1 \\
        1 \div 2 = 0\ R&1
    \end{align*}
    The number $(100)_{10}$ in 8 bit unsigned and Two's Complement notation
    would be
    $\boxed{01100100.}$ \\
    6:
    \par Using the same technique as before and the same principle,
    \begin{align*}
        6 \div 2 = 3\ R&0 \\
        3 \div 2 = 1\ R&1 \\
        1 \div 2 = 0\ R&1
    \end{align*}
    the 8 bit representation for the decimal number 6 in Two's Complement as
    well as unsigned integers, would be $\boxed{00000110.}$ \\
    -5:
    \par Since this number is negative, it is signed and therefore cannot be
    represented using the unsigned integers. In this case, the number must be
    converted to binary and then using Two's Complement, made negative.
    \begin{align*}
        5 \div 2 = 2\ R&1 \\
        2 \div 2 = 1\ R&0 \\
        1 \div 2 = 0\ R&1
    \end{align*}
    Taking this number in 8 bits, $(00000101)_{2}$ and converting it to negative
    using Two's Complement, first we flip all the bits yielding,
    $(11111010)_{2}$ and then we add 1 to the least significant bit,
    $\boxed{11111011.}$
    \begin{itemize}
        \item[3.] How do you represent 264 and -264 (assume the numbers are 8
            bits)?
    \end{itemize}
    \par Finding the range for the maximum possible integer for 8 bit numbers,
    for unsigned we find that it is from 0 to $2^{8} - 1 = 255$, and for Two's
    Complement it ranges from $-2^{8-1}$ to $2^{8-1} - 1$ which is from -128 to
    127. In both of these cases, it can be seen that the value of 264 resides
    outside of the range for 8 bit numbers and therefore cannot be represented
    using either notation.
    \begin{itemize}
        \item[4.] What is the range of decimal (presented with powers of 2) that
            32-bit Two's Complement can represent? Please explain why.
    \end{itemize}
    \par The range of numbers would be from $-2^{32-1}$ to $2^{32-1}-1$, which
    would result in a range from -2147483648 to 2147483647. Represented using
    powers of 2 this would be, from -2 Gi to 2 Gi - 1.  This happens to be the
    range since Two's Complement has the ability to represent both positive and
    negative numbers and adding two opposite numbers together would have to
    result in zero. For this reason, since there are only a limited number of
    bits, in this case 32, the range will be half that of unsigned 32-bit
    integers. The maximum unsigned 32-bit integer can be represented as $2^{32}$
    which is 4294967296. This number is cut in half, and the negative integers
    can be represented up to -2147483648, and the maximum positive integer will
    be 2147483647 since we have to account for zero.
    \begin{itemize}
        \item[5.] Why do computers use bytes for address representation?
    \end{itemize}
    \par Using bits would likely be too costly to be beneficial. Bytes are used
    for addressing as this allows for more addresses to be defined with a large
    number of bits.
    \begin{itemize}
        \item[6.] How much data can we store on a 8 K-byte memory chip?
    \end{itemize}
    \par 8 bytes will be equal to 8000 bytes, and since there are 8 bits in one
    byte, there are 64000 bits that can be stored on the chip. Since each of the
    registers in the chip can hold 32 bits, this leaves 2000 addresses in which
    information can be stored.
    \section{Counting}
    \begin{itemize}
        \item[1.] How many bits do we need to represent a variable that can only
            take on the values 10, 20, 30, or 40?
    \end{itemize}
    \par Since the variable can only have four possible values, $2^{2}$, 2 bits
    would be required to cover the four states.
    \begin{itemize}
        \item[2.] If the only value a variable can take on is 98.65, how many
            bits are needed to represent it?
    \end{itemize}
    \par Since there is only one value for this variable to take on the lowest
    possible number of bits would be 1 bit since it can hold two values and we
    cannot go lower than that.
    \begin{itemize}
        \item[3.] If we need to address 4 Mi-Byte of memory and we want to
            address every byte of memory, how long does an address need to be?
    \end{itemize}
    \par 4 Mi-Bytes would be equal to $4 * 2^{20}$ Bytes, which equals $2^{22}$
    Bytes. Using the equation $\log_2(n)$ which gives the number of bits needed
    to address some $n$ number of bytes, we get that 22 bits are needed for each
    address.
\end{document}
