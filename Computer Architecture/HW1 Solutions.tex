\documentclass[12pt]{article}
\usepackage{indentfirst}
\usepackage{amsmath}
\setlength{\jot}{2ex}
\usepackage{mathrsfs}
\usepackage{graphicx}
\usepackage{wrapfig}
\usepackage{booktabs}
\usepackage[letterpaper, margin=1in]{geometry}
\usepackage{fancyhdr}
\pagestyle{fancy}
\fancyhead[R]{Homework 1}
\fancyfoot[C]{\thepage}
\renewcommand{\headrulewidth}{1pt}
\renewcommand{\footrulewidth}{1pt}
\usepackage [autostyle, english = american]{csquotes}
\MakeOuterQuote{"}
\usepackage{enumitem}
\renewcommand{\baselinestretch}{1.0}
\newcommand{\objects}[2]{%
  \leavevmode\vbox{\hbox{#1}\nointerlineskip\hbox{#2}}%
}
\begin{document}
    \section*{Problem 1}
    \begin{enumerate}
        \item[(1)]
            Personal computer - Macbook \\
            Personal mobile device - iPhone \\
            Embedded computer - Navigation system within car \\
            Server computer - Video game multiplayer servers \\
            Supercomputer - IBM Roadrunner supercomputer \\
            Warehouse computer (Cloud computing) - Amazon Web Services (EC2 \&
            S3)
        \item[(2)]
            \begin{itemize}[leftmargin=*]
                \item[-] Personal computer: This device must be able to handle a
                    multitude of tasks at the same time and be relatively silent
                    and therefore energy efficient as they will be connected to
                    the home and the user at all times.
                \item[-] Personal mobile device: This device must be portable
                    and be able to withstand impacts and motion without damage
                    to the internal processors and battery.
                \item[-] Embedded computer: This computer must be able store a
                    vast amount of information as it will be working offline
                    most of the time and therefore needs to have the data it
                    needs to function on hand at all times.
                \item[-] Server computer: Servers are usually very big and are
                    required to be able to handle a lot of traffic as people are
                    logging in and out constantly. The processors have to be
                    fast and the energy constraint is not much of a problem.
                \item[-] Supercomputer: These computers are used for massive
                    tasks such as scientific simulations and large computations
                    therefore,  they require immense processing power as well as
                    a large amount of storage space due to their large
                    collection  of data.
                \item[-] Warehouse computer: These computers need to be
                    encrypted as they handle a large amount of sensitive
                    information and this also requires a large amount of storage
                    space for that data.
            \end{itemize}
        \item[(3)]
            One computer that is not listed above in any of the categories but
            would be fitting, is the calculator. This would be an example of a
            mobile computer since it provides one function and for the most
            part, they are portable. The calculator, or even a simple adding
            machine, was one of the first computers to exist for the sole
            purpose of dealing with numbers. For this reason, it would be an
            ideal example of a mobile computer.
    \end{enumerate}
    \section*{Problem 2}
    \begin{enumerate}
        \item[(1)]
            The five components of computer architecture are, input, output,
            memory, datapath, and control. These five components come together
            to form every computer. Every computer contains parts from each of
            these five categories and this allows for it to function properly.
            The input serves the basic function of supplying the computer with
            information such as the task which needs to be done as well as any
            other givens. The datapath and control form the processor which
            actually completes the task, with the help of the memory. The
            memory's function is to supply the processor with the needed data
            and instructions in order to complete the task. Finally comes the
            output which is the result of the processor's computation.
        \item[(2)]
            A smartphone is an example of a computer, which implies that its
            parts will fall into each of the five categories. The input and
            output can be said to be the touchscreen display of the phone as
            this is what the user interacts with and through which gives the
            computer its tasks. The 4 gigabytes of memory and the processor can
            be found to be in Apple's A12 chip, this is where the memory,
            datapath, and control categories can be found.
        \item[(3)]
            If we take personal computer for example, the parts that make that
            type of computer up are relatively the same however, differ in many
            ways. For a large scale PC, the inputs and outputs are no longer
            just a screen. Inputs are given by external peripherals such as a
            keyboard and a mouse, and the output is seen through an external
            monitor. The memory as well on these computers is not the same since
            on a PC, the sticks of RAM are not integrated directly on the
            processor rather, they are connected to the motherboard and are of
            larger size. Up to 64 gigabytes is valid compared to the lowly 4 on
            the smartphone. The processor, datapath and control, is also
            designed for much higher intensity work and has tens of cores
            whereas a smartphone may only have a few. Essentially, each of the
            parts performs the same function, but the integration of each of
            them is vastly different.
    \end{enumerate}
    \section*{Problem 3}
    \begin{enumerate}
        \item[(1)]
            \[
                \text{CPU time} = \frac{\text{Instruction Count} \times
                \text{CPI}}{\text{Clock Rate}}
            \]
            \begin{itemize}[leftmargin=*]
                    \item[-] Processor A:
                    \[
                        \text{CPU Time} = \frac{(0.4)(1)+(0.2)(2)+(0.4)(6)}{1
                        \times 10^{9}} = 3.2 \times 10^{-9}\ s
                    \]
                    \item[-] Processor B:
                    \[
                        \text{CPU Time} = \frac{(0.4)(1.5)+(0.2)(3)+(0.4)(7)}{2
                        \times 10^{9}} = 2 \times 10^{-9}\ s
                    \] \,
                    \item[]
                    This shows that Processor B is faster than Processor A since
                    it has a shorter CPU time. To determine how much faster, the
                    CPU time of the first processor can be divided by the time
                    of the second, $\frac{3.2 \times 10^{-9}}{2 \times
                    10^{-9}}$, which gives a value of $1.6$ times faster.
            \end{itemize}
        \item[(2)]
            \[
                \text{Amdahl's Law} =
                \frac{T_{\text{affected}}}{\text{Improvement Factor}} +
                T_{\text{unaffected}}
            \]
            \[
                \text{Improvement Factor} = 2
            \]
            \[
                T_{\text{affected}} = \frac{(0.2)\text{(Branch Instruction
                CPI)}}{\text{CPU Clock Rate}}
            \]
            \[
                T_{\text{unaffected}} = \frac{(0.4)\text{(ALU CPI)} +
                (0.4)\text{(Memory Load CPI)}}{\text{CPU Clock Rate}}
            \]
            \begin{itemize}
                \item[-] Processor A:
                    \[
                        \text{CPU time} = \frac{1}{2} \times \frac{(0.2)(2)}{1
                        \times 10^{9}} + \frac{(0.4)(1) + (0.4)(6)}{1 \times
                        10^{9}} = 3 \times 10^{-9}\ s
                    \]
                    \[
                        \frac{\text{(CPU Time)}_{1}}{\text{(CPU Time)}_{2}} =
                        \frac{3.2 \times 10^{-9}}{3 \times 10^{-9}}= 1.0667
                    \]
                \item[-] Processor B:
                    \[
                        \text{CPU time} = \frac{1}{2} \times \frac{(0.2)(3)}{2
                        \times 10^{9}} + \frac{(0.4)(1.5) + (0.4)(7)}{2 \times
                        10^{9}} = 1.85 \times 10^{-9}\ s
                    \]
                    \[
                        \frac{\text{(CPU Time)}_{1}}{\text{(CPU Time)}_{2}} =
                        \frac{2 \times 10^{-9}}{1.85 \times 10^{-9}}= 1.0811
                    \]
                From the calculations above, it can be seen that with the
                acceleration of the performance of the branch instructions,
                Processor A is about 1.067 times faster, and Processor B is
                about 1.08 times faster than the speed without the
                acceleration.
            \end{itemize}
    \end{enumerate}
    \section*{Problem 4}
    \begin{enumerate}
        \item[(1)]
            Moore's Law states that integrated circuit resources will double
            every two years.
        \item[(2)]
            The textbook likely removed the idea of "designing with Moore's Law
            in mind" due to the fact that it is not a foundational construct of
            computer architecture any longer. "No exponential growth can last
            forever," and this is true for Moore's Law as well since it can be
            seen to be slowing down as time progresses. Integrated circuits are
            not doubling in power and performance every two years and for this
            reason, it was likely taken out of the $2^{nd}$ edition of the book.
        \item[(3)]
            The fact the Moore's Law is no longer valid, provides implications
            of stagnation within the advancement of computer processors and
            memory. Possible solutions may only come in the form of vastly
            different ways of making transistors and computer parts in general.
            As the end of the Law nears, the traditional transistor will likely
            have to be replaced by a new form of technology, such as quantum
            mechanical transistors. This would be the most extreme form of
            multiprocessing as quantum computers can evaluate far more tasks
            simultaneously than traditional computers with any amount of cores.
    \end{enumerate}
\end{document}
