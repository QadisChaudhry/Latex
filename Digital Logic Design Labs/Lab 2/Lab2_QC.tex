\documentclass[12pt]{article}
\usepackage{amsmath}
\usepackage{askmaps}
\usepackage{graphicx}
\usepackage{wrapfig}
\usepackage{booktabs}
\usepackage{tikz-timing}
\usepackage[letterpaper, margin=1in]{geometry}
\renewcommand{\baselinestretch}{1.0}
\title{Lab 2 \\ Combinational SSI Circuits}
\author{Qadis Chaudhry}
\date{March 12, 2021}
\newcommand{\twoobjects}[2]{%
  \leavevmode\vbox{\hbox{#1}\nointerlineskip\hbox{#2}}%
}
\begin{document}
\maketitle
    \section*{Experiment 1}
    \begin{itemize}
        \item[\textit{i)}]
    \end{itemize}
    \[
        F\left(A,B,C,D\right) = \Sigma_{A,B,C,D}\left(1,3,5,6,7,14\right)
    \]
    \begin{center}
        \begin{tabular}{ccccc|c}
            Minterm & A & B & C & D & F \\
            \cmidrule{2-6}
            0 & 0 & 0 & 0 & 0 & 0 \\
            1 & 0 & 0 & 0 & 1 & 1 \\
            2 & 0 & 0 & 1 & 0 & 0 \\
            3 & 0 & 0 & 1 & 1 & 1 \\
            4 & 0 & 1 & 0 & 0 & 0 \\
            5 & 0 & 1 & 0 & 1 & 1 \\
            6 & 0 & 1 & 1 & 0 & 1 \\
            7 & 0 & 1 & 1 & 1 & 1 \\
        \end{tabular}
        \quad
        \begin{tabular}{ccccc|c}
            & A & B & C & D & F \\
            \cmidrule{2-6}
            8 & 1 & 0 & 0 & 0 & 0 \\
            9 & 1 & 0 & 0 & 1 & 0 \\
            10 & 1 & 0 & 1 & 0 & 0 \\
            11 & 1 & 0 & 1 & 1 & 0 \\
            12 & 1 & 1 & 0 & 0 & 0 \\
            13 & 1 & 1 & 0 & 1 & 0 \\
            14 & 1 & 1 & 1 & 0 & 1 \\
            15 & 1 & 1 & 1 & 1 & 0 \\
        \end{tabular}
    \end{center}
    \par This truth table represents the function shown above and it can be seen
    that the function is defined as an expression of Minterms instead of a
    Boolean expression. Minterms by definition are terms for which the output of
    the function equals 1. Using this knowledge, the truth table can be filled
    out according to the positions of the Minterms on the table. Since they
    equal to 1, the table values for 1, 3, 5, 6, 7, and 14, are set to equal 1,
    while all the other terms are set to 0. The result of this, is the table
    shown above. From this information, a Karnaugh Map can also be constructed
    to described the function.
    \begin{center}
        \askmapiv{}{CDAB}{F}{0000110001011100}{}
    \end{center}
    \par The K-Map shown above is the representation of the function that was
    giving at the beginning of the experiment. The main purpose of a Karnaugh
    Map is for the simplification of Boolean expressions and this is done by
    combining adjacent 1s that appear in the map. The map itself is constructed
    using the list of Minterms. In a given case where the Minterms are not
    explicitly defined, contrary to this case, the Minterms can be extracted
    from the truth table of the function using the indexing that is shown above
    in the table. These Minterms are placed in specific cells of the map which
    are defined as, reading left to right and top to bottom, $0, 1, 3, 2, 4, 5,
    7, 6, 12, 13, 15, 14, 8, 9, 11, 10$. With this information, we can place 1s
    where they need to be placed, the cell which corresponds to the Minterm. For
    this particular function, the cells $1, 3, 5, 6, 7$, and $14$ will be filled
    in with 1s and the rest of the cells will be filled with 0s. Doing this will
    yield the map above.
    \par The main purpose of a Karnaugh Map as stated before, is for the
    simplification of Boolean expressions. This can be done by grouping together
    adjacent 1s that are in couples equal to a power of two. This means we can
    create groups that are either two 1s, four 1s, eight 1s, and so on,
    otherwise, a group cannot be created. These groups are called prime
    implicants.  Another aspect that must be adhered to is the creation of as
    least groups as possible, since this will result in the simplest form of the
    function. In this map, it can be seen that there is a group of 1s occupying
    cells $1, 3, 5$, and $7$.  These can be grouped and should be grouped, since
    they are the largest power of two number of 1s that are adjacent in the
    Karnaugh Map. Now looking to see what is left, the only 1s that are not
    covered now are the ones in cells $6$ and $14$. These two can be grouped
    together and the resulting map is formed.
    \begin{center}
        \askmapiv{}{CDAB}{F}{0000110001011100}{%
            \put(2.0,3.0){\oval(1.8,1.8)}%
            \put(3.5,2.0){\oval(0.8,1.8)}
        }
    \end{center}
    \par With the use of this map, the function can now be simplified by
    logically defining the essential prime implicants that are formed. Usually,
    when forming the groups in a Karnaugh Map, all adjacent 1s are grouped
    together according to powers of two, in this case however, it can be seen
    that there is a pair of adjacent 1s in the cells $6$ and $7$ that are not
    grouped. The reason for this lies in the goal of creating the simplest
    function that can be made, meaning, the least amount of groups are formed.
    The implications of this will be explored in the next section of this
    experiment however, for now, the simplified function is attained.
    \par In order to determine this simplified expression, we must assign logic
    values based on the variables to the groups created in the map. For the
    group of four, it can be examined that the bit values of one of the
    variables is static while the other is not. For this group, the bit values
    corresponding to the variables $A$ and $B$ change from $00$ and $01$
    respectively. The value of $A$ remains constant while the value of $B$
    changes from $0$ to $1$. Since the value of $B$ is changing while the output
    of the function is $1$, it can be intuitively reasoned that the output of
    the function corresponding to that Minterm, does not depend on the variable
    $B$ but does depend on $A$. The same can be done for the variables $C$ and
    $D$ and it can be seen that for these variables, the bit values change from
    $01$ to $11$ where, $D$ is the one that is static while $C$ is changing.
    With the same reasoning, the output can be said to be dependent on the value
    of $D$, and independent of the variable $C$. Writing this as a complete
    Boolean statement, since $A$ is static and has a value of $0$, and $D$ is
    static and has a value of $1$, it can be said that the value of the output
    is dependent on the inverse of the variable $A$, and the input value of the
    variable $D$. This can be written as a logic function, NOT $A$ AND $D$, or
    $A'D$. This covers the group of four in the Karnaugh Map and therefore the
    four Minterms in the function that corresponded to those cells, now we can
    move on to the other group.
    \par The only other group in the map is the group of two 1s in cells, $6$
    and $14$. Using the same approach to find an expression for this group, we
    can see that the bit values $C$ and $D$ remain static since the implicant
    resides only in one column, but the values for $A$ and $B$ are changing. For
    this reason, the output depends both on $C$ and $D$ fundamentally since the
    values for both do not change. For the variables $A$ and $B$ however, the
    bits go from $01$ to $11$. The value of $A$ here is not static as it changes
    from $0$ to $1$ while the value of $B$ stays at $1$. This suggests, as
    before, that the output of the function for these cells depends on the value
    of $B$ and not on the value of $A$. Altogether, this group can be described
    as, $B$ AND $C$ AND NOT $D$, or simply $BCD'$. With both of these prime
    implicants now defined, the expression of the function can be written.
    \begin{equation}
        F\left(A,B,C,D \right) = A'D + BCD'
    \end{equation}
    \par Since all Boolean functions can be represented as a Sum of Products,
    and we have the products that define the two groups created in the Karnaugh
    Map, simply adding these products together, taking them as the input of an
    OR gate, will grant an expression for the function. This will be the Minimal
    Sum of Products expression since it has been made the simplest with the use
    of the Karnaugh Map.
    \begin{itemize}
        \item[\textit{ii)}]
    \end{itemize}
    \begin{figure}[h]
        \centering
        \includegraphics[width=0.6\linewidth]{Hazardous Circuit Implementation
        (Experiment 1).png}
        \caption{Circuit Implementation, Eq. 1}%
        \label{fig:1}
    \end{figure}
    \par This circuit is an accurate realization of the function defined above,
    two AND gates representing the two prime implicants, and their outputs being
    taken as the input of an OR gate to yield the expression. However, the
    circuit consists of timing hazard notably, static-1 hazard. In real world
    circuits, the gates which make up the circuit are not ideal, they possess
    intrinsic delays due to a multitude of factors. This causes glitches in the
    output of the function as these delays cause there to be differences in
    timing of the output of each gate. These delays are known as propagation
    delays since they propagate to the final output of the function and cause
    glitches. The glitches are categorized as three different types, static-1
    hazard, static-0 hazard, and dynamic hazard. Static-1 hazard is classified
    by the behavior of the function dipping down to a value of zero when the
    value should be static, at 1. The observed change in the output should be
    nonexistent.  Similarly, static-0 hazard occurs when there is a change in
    the output from the value of zero. Since no change should exist at all but
    the function changes to 1 for a brief moment, the circuit possesses static-0
    hazard.  Dynamic hazard is simply consistent of multiple static-1 and
    static-0 hazards and is classified by multiple glitches in the output. In
    this particular case, there exists a static-1 hazard and it can be shown
    with the use of a timing diagram.
    \begin{center}
        \begin{tikztimingtable}
            A & 25L \\
            B & 25H \\
            C & 25H \\
            D & 7H 18L \\
            AP & 25H \\
            DP & 9L 16H \\
            DAP & 9H 16L \\
            BCDP & 11L 14H \\
            F & 9H 2L 14H \\
            \begin{extracode}
                \begin{background}
                    \draw[->, >=latex] (0, -\nrows-8) -- (\twidth+1, -\nrows-8);
                    \foreach\n in {0,1,...,\twidth}
                        \draw(\n,-\nrows-8+.2) -- + (0,-.4)
                            node [below,inner sep=2pt] {\scalebox{1}{\tiny\n}};
                \end{background}
            \end{extracode}
        \end{tikztimingtable}
    \end{center}
    \par The scale of this timing diagram is from 0 to 25 where each step is
    divided to equal $5ns$. It shows that the delay between the gates is $10ns$,
    and depicts these delays as they occur with every gate as well as how they
    propagate to the output in the end as well. It can be seen that the output
    of function consists a glitch where the value of $F$ changes from 1 to 0 for
    a brief moment before returning to 1. This is described as a static-1 hazard
    since the output should have remained at 1 however, fluctuated down to 0. In
    the problem it states that the clock timing is $100\mu s$, however, for this
    diagram, the scale has been chosen so that only one change is in the input
    of $D$ is shown. This was done so that the change in the output could be
    shown clearly as the duration of the glitch is a mere $10ns$.
    \begin{itemize}
        \item[\textit{iii)}]
    \end{itemize}
    \par One method of detecting and fixing these glitches is to add buffer
    gates to the circuit. This has the effect of equalizing the propagation
    delays between all the gates and the output will exhibit an accurate reading
    based on its inputs. The issue with this, however, is that we are adding
    more and more gates to the circuit, effectively slowing it down which is not
    an ideal solution. The main method that is utilized for the purpose of
    eliminating hazards, is with the use of a Karnaugh Map. This time, instead
    of using it to find the simplest function, we will use it to eliminate
    hazard.
    \par To eliminate hazard with a Karnaugh Map, we must look for adjacent 1s
    that are not coupled together in the case of static-1 hazard. Since the
    simplest function looks to create the smallest number of groups, it neglects
    these pairs of redundant 1s, which in turn leads to a simpler function
    however, one that is prone to error. We can look at the Karnaugh Map created
    previously and modify it to get rid of the hazard.
    \begin{center}
        \askmapiv{}{CDAB}{F}{0000110001011100}{%
            \put(2.0,3.0){\oval(1.8,1.8)}%
            \put(3.5,2.0){\oval(0.8,1.8)}
        }
    \end{center}
    \par This map covers all the 1s in the least number of groups and therefore
    creates the simplest function. For the use of eliminating hazard, we must
    look to the pair of adjacent 1s that are not covered by a group in cells $6$
    and $7$. Since the mitigation of error requires the addition of redundancy,
    the implicant of these two ones must be added into the function. The new
    Karnaugh Map will take the form:
    \begin{center}
        \askmapiv{}{CDAB}{F}{0000110001011100}{%
            \put(2.0,3.0){\oval(1.8,1.8)}%
            \put(3.0,2.5){\oval(1.8,0.8)}%
            \put(3.5,2.0){\oval(0.8,1.8)}
        }
    \end{center}
    \par Defining this implicant, we have to once again look at the variables
    that are changing and the variables that are staying constant. For $A$ and
    $B$, we observe no change since the implicant resides in a single row
    therefore there are no changes in the bit values that define $A$ and $B$.
    For $C$ and $D$ however, the bit values do change; from, $11$ to $10$.  Once
    again, here it can be observed that $C$ remains constant while the bit value
    of $D$ changes between $1$ and $0$. This means that the output of that group
    is dependent on the value of $C$, and $A$ and $B$, while independent of $D$.
    Writing this as a logical expression, this implicant in dependent on,
    $A'BC$. Adding this on to the function of the output, a new function can be
    defined as:
    \begin{equation}
        G\left(A,B,C,D \right) = A'D + BCD' + A'BC
    \end{equation}
    \\
    \begin{itemize}
        \item[\textit{iv)}]
    \end{itemize}
    \begin{figure}[h]
        \centering
        \includegraphics[width=0.5\linewidth]{Fixed Circuit Implementation
        (Experiment 1).png}
        \caption{Circuit Implementation, Eq. 2}%
        \label{fig:2}
    \end{figure}
    \par This updated circuit is of the function depicted in equation 2 and it
    can be seen that it only differs slightly from the original circuit. The
    only difference is that this circuit consists of one extra gate, $BCAP:
    BCA'$. This intentional redundancy allows for the function to operate hazard
    free and this can be shown with the use of a timing diagram. This is the
    same as before, however, now we can observe that there is no glitch present
    in the output of the function.
    \begin{center}
        \begin{tikztimingtable}
            A & 25L \\
            B & 25H \\
            C & 25H \\
            D & 7H 18L \\
            AP & 25H \\
            DP & 9L 16H \\
            DAP & 9H 16L \\
            BCDP & 11L 14H \\
            BCAP & 25H \\
            G & 25H \\
            \begin{extracode}
                \begin{background}
                    \draw[->, >=latex] (0, -\nrows-9) -- (\twidth+1, -\nrows-9);
                    \foreach\n in {0,1,...,\twidth}
                        \draw(\n,-\nrows-9+.2) -- + (0,-.4)
                            node [below,inner sep=2pt] {\scalebox{1}{\tiny\n}};
                \end{background}
            \end{extracode}
        \end{tikztimingtable}
    \end{center}
    \par It can be seen that the output of the function is constant with a value
    of $1$. This is the expected result as the output for both bit values $0111$
    and $0110$ are 1, referring back to the truth table created for the
    function. The scale once again is from 0 to 25 where all each division
    represents a value of $5ns$. The previous behavior was not concurrent with
    the expected behavior as the function assumed a value of 0 for $10ns$.  This
    was the glitch present in the preliminary implementation of the function and
    it is clear that it has been mitigated with the creation of the new
    function. The way this is accomplished is quite simple, since the function
    is a Sum of Products, the output is the result of an OR gate. In this case,
    it entails that the function value is orchestrated by the value of the gate
    $BCA'$, and since in this case the value of that gate is always 1 for this
    transition, as is the output of the function. This method of mitigation is
    not only useful for this case but any case of the function's input
    information changing. The reason for this is, looking back to the Karnaugh
    Map, there are no adjacent 1s present without a grouping. All possible
    changes to the input bit values are covered by an individual gate, and since
    these gates are all the eventual input of an OR gate, the function is
    covered for all possible changes to the values of its inputs.
    \section*{Experiment 2}
    \begin{itemize}
        \item[\textit{i)}]
    \end{itemize}
    \begin{center}
        \twoobjects
            {\includegraphics[width=0.7\linewidth]{Experiment 2 (A=0) 2021-02-26
            at 2.16.27 PM.png}}
            {\includegraphics[width=0.7\linewidth]{Experiment 2 (A=1) 2021-02-26
            at 2.15.54 PM.png}}
    \end{center}
    \par To realize the circuit for equation 2, the Emona system is used. Here,
    we are able to create a circuit that develops the output of the function
    using real world gates and is able to depict what type of behavior is
    expected from the function. For this implementation, we require one AND gate
    with two inputs, two AND gates with three inputs, two inverters, and one OR
    gate. The inverters will be used for the inversion of the inputs $A$ and
    $D$, the two input AND gate will take the input of $D$ and the inverted
    input of $A$, and the two three input AND gates will both take the input of
    $BC$ and one will take the input of $A'$ while the other, $D'$. The OR gate
    will be placed at the end, with its input being the outputs of the three AND
    gates. The resulting output will be function $G\left(A,B,C,D\right)$.
    \par One thing that we must take into account when developing the waveforms
    for this function is that it is a function of four variables. Since Emona
    only possesses four monitoring channels, one of the inputs will have to be
    controlled since it cannot be monitored. In the Emona implementation, we can
    see that the value of one of the variables is controlled using a manual
    switch. This means that the value of that variable is set as either 1 or 0
    while the other three variables can be set on a clock and monitored with the
    use of one of the channels. The output then is able to be monitored by the
    last channel. This is the reason there are two images for the circuit.  The
    variable $A$ was chosen to be controlled by the switch in this experiment,
    while $B$, $C$, and $D$ were monitored with ChA, ChB, and ChC respectively.
    The output was set to be monitored by ChD. In the first image, the value of
    $A$ is set to be 0 and in the second image $A$ equals 1. The truth table can
    be used to verify that the values of the two functions, equation 1 and
    equation 2, are the same, the only difference being, one is hazardous while
    the other is hazard free.
    \[
        G\left(A,B,C,D \right) = A'D + BCD' + A'BC
    \]
    \begin{center}
        \begin{tabular}{ccccc|c}
            Minterm & A & B & C & D & G \\
            \cmidrule{2-6}
            0 & 0 & 0 & 0 & 0 & 0 \\
            1 & 0 & 0 & 0 & 1 & 1 \\
            2 & 0 & 0 & 1 & 0 & 0 \\
            3 & 0 & 0 & 1 & 1 & 1 \\
            4 & 0 & 1 & 0 & 0 & 0 \\
            5 & 0 & 1 & 0 & 1 & 1 \\
            6 & 0 & 1 & 1 & 0 & 1 \\
            7 & 0 & 1 & 1 & 1 & 1 \\
        \end{tabular}
        \quad
        \begin{tabular}{ccccc|c}
            & A & B & C & D & G \\
            \cmidrule{2-6}
            8 & 1 & 0 & 0 & 0 & 0 \\
            9 & 1 & 0 & 0 & 1 & 0 \\
            10 & 1 & 0 & 1 & 0 & 0 \\
            11 & 1 & 0 & 1 & 1 & 0 \\
            12 & 1 & 1 & 0 & 0 & 0 \\
            13 & 1 & 1 & 0 & 1 & 0 \\
            14 & 1 & 1 & 1 & 0 & 1 \\
            15 & 1 & 1 & 1 & 1 & 0 \\
        \end{tabular}
    \end{center}
    \par It can be seen here that the truth table for this function and the one
    created at the beginning of experiment 1 are the same. This signifies, that
    the two functions are indeed equivalent. In this particular case, since the
    table is formatted so that $A=0$ on the left and $A=1$ on the right, the
    table on the left corresponds to the first image of the circuit and its
    waveform while the table to the right represents the second image as well as
    its waveform. Although the table was split like this to avoid creating one
    large vertical table, in this case, it serves functional importance as it
    depicts the Minterms that correspond to each of the values of $A$ according
    to the switch.
    \section*{Experiment 3}
    \begin{itemize}
        \item[\textit{i)}]
    \end{itemize}
    \begin{center}
        \twoobjects
            {\includegraphics[width=0.85\linewidth]{Experiment 3 Original (A=0)
            2021-02-26 at 2.55.58 PM.png}}
            {\includegraphics[width=0.85\linewidth]{Experiment 3 Original (A=1)
            2021-02-26 at 2.56.07 PM.png}}
    \end{center}
    \par For this circuit, the implementation can be simplified greatly and that
    is evident from looking at the function's Boolean expression as well as its
    truth table.
    \begin{figure}[h]
        \centering
        \includegraphics[width=0.65\linewidth]{Original Circuit Schematic
        (Experiment 3).png}
        \caption{Circuit Implementation of $H$}%
    \end{figure}
    \par Using this diagram for the circuit, we can write an expression that
    describes the output. This expression can then be simplified using the
    axioms and theorems of Boolean Algebra. Working from the top, it can be seen
    that the input $D$ is being used as the input of an OR gate. The other input
    of this OR gate is the output of the NAND gate which requires the variables
    $A$ and $B$. Looking at $A$, the variable is being inverted and then taken
    as the input of a NAND gate. The other input to this NAND gate is the
    unaltered value of $B$. The expression for this NAND gate will be simply,
    $(A'B)'$, since the function of the NAND gate is equivalent to the inverted
    output of an AND gate. The output of this gate is then taken as the input of
    the OR gate with the variable $D$. This results in the output of the OR gate
    being $(A'B)'+D$. This is one of the inputs to the final NAND gate that
    gives the result of the function, the other comes from the NOR gate seen at
    the bottom. This NOR gate takes in the values of $B$ and $C$. Since the
    function of a NOR gate is equal to that of an OR gate with an inverter at
    its output, the output of this gate will be $(B+C)'$.  Since the output of
    the function is given as the NAND of these two outputs, we can AND the
    output of the OR gate with the output of the NOR gate and, invert the output
    of that, to attain the equation for the function $H$.
    \begin{equation}
        H\left(A,B,C,D\right) = ((B+C)'((A'B)'+D))'
    \end{equation}
    \par The Emona circuits shown above are the implementation of this function,
    shown in equation 3.  Once again, there are four inputs to this function,
    requiring one of the variables of the function to be manually controlled by
    a switch. The others are clocked and monitored as is the output. This will
    later be compared to the simplified function to gauge whether or not the
    simplification is correct.
    \par One method of simplification of a Boolean function is through the use
    of a Karnaugh Map, which was used to create the function shown in equation
    1. This time however, since we have the function from the circuit diagram,
    we can employ the laws of Boolean Algebra in order to create a simpler
    expression.
    \newpage
    \textit{Steps of Simplification.}
    \begin{align*}
        ((B+C)'((A'B)'+D))' & = ((B+C)')'+((A'B)'+D)' & (1) \\
                            & = (B+C)+((A'B)'+D)'     & (2) \\
                            & = (B+C)+((A'B)')'(D)'   & (3) \\
                            & = (B+C)+(A'B)D'         & (4) \\
                            & = B+C+A'BD'             & (5) \\
                            & = B+BA'D'+C             & (6) \\
                            & = B(1+A'D')+C           & (7) \\
                            & = B(1)+C                & (8) \\
        ((B+C)'((A'B)'+D))' & = B+C                   & (9)
    \end{align*}
    \par These steps are shown for the conversion of the original expression
    into its simplest form using Boolean Algebra. The process involves the use
    of the different theorems which are used to write equivalent expressions.
    For the first step, we apply the DeMorgan's Law which inverts both of the
    terms of the expression and switches the operation acting between them. From
    this we get the resulting function in line 2, and it can be seen that $B+C$
    is inversed twice. By the Double Negation Law, this simply equates to $B+C$.
    From here, we can apply the DeMorgan's Law once again to the second term,
    getting us from line 2 to 3. The Double Negation Law is present once again
    with the term $A'B$, giving us back that term, unaltered. The general
    function is written in line 5 with nothing more that can be done using the
    DeMorgan's Law however, the function can be simplified further. Rearranging
    the terms to get the form in line 6, we are able to use the Distributive
    Property to rewrite the two terms on the left. The $B$ can be factored out
    in a sense and line 7 in attained. With this, and the simple fact that
    anything ORed with 1 will simply yield 1, we can write the function as line
    8. Another axiom that can be employed now is that anything ANDed with 1 will
    depend only on the variable, $B(1)=B$. This gets us to the completely
    simplified expression written in line 9. This can be implemented with the
    use of only one gate and two variables:
    \begin{figure}[h]
        \centering
        \includegraphics[width=0.85\linewidth]{Simplified Circuit Schematic
        (Experiment 3).png}
        \caption{Simplified Circuit Implementation of $H$}%
    \end{figure}
    \par This can also be implemented using the Emona Board and the inputs and
    outputs can be monitored. The wave forms for this circuit can then be
    compared to that of the original circuit and equality can be verified.
    \begin{figure}[h]
        \centering
        \includegraphics[width=0.85\linewidth]{Experiment 3 Simplified
        2021-02-26 at 2.28.16 PM.png}
        \caption{Simplified Emona Implementation}%
    \end{figure}
    \par From this, we can see that the inputs are monitored with channel ChA
    and ChB and the output with ChC. The waveform created by ChC can now be
    compared to the output seen by ChD in the original implementation of the
    function. Doing this, we can see that the waveforms are identical and
    therefore, the circuits have to be as well. Furthermore, to truly ensure the
    equality of these two circuits, the truth tables for each of the functions
    can be compared.
    \[
        H\left(A,B,C,D\right) = ((B+C)'((A'B)'+D))'
    \]
    \begin{center}
        \begin{tabular}{ccccc|c}
            Minterm & A & B & C & D & H \\
            \cmidrule{2-6}
            0 & 0 & 0 & 0 & 0 & 0 \\
            1 & 0 & 0 & 0 & 1 & 0 \\
            2 & 0 & 0 & 1 & 0 & 1 \\
            3 & 0 & 0 & 1 & 1 & 1 \\
            4 & 0 & 1 & 0 & 0 & 1 \\
            5 & 0 & 1 & 0 & 1 & 1 \\
            6 & 0 & 1 & 1 & 0 & 1 \\
            7 & 0 & 1 & 1 & 1 & 1 \\
        \end{tabular}
        \quad
        \begin{tabular}{ccccc|c}
            & A & B & C & D & H \\
            \cmidrule{2-6}
            8 & 1 & 0 & 0 & 0 & 0 \\
            9 & 1 & 0 & 0 & 1 & 0 \\
            10 & 1 & 0 & 1 & 0 & 1 \\
            11 & 1 & 0 & 1 & 1 & 1 \\
            12 & 1 & 1 & 0 & 0 & 1 \\
            13 & 1 & 1 & 0 & 1 & 1 \\
            14 & 1 & 1 & 1 & 0 & 1 \\
            15 & 1 & 1 & 1 & 1 & 1 \\
        \end{tabular}
    \end{center}
    \[
        H_s\left(A,B,C,D\right) = B+C
    \]
    \begin{center}
        \begin{tabular}{ccccc|c}
            Minterm & A & B & C & D & $H_s$ \\
            \cmidrule{2-6}
            0 & 0 & 0 & 0 & 0 & 0 \\
            1 & 0 & 0 & 0 & 1 & 0 \\
            2 & 0 & 0 & 1 & 0 & 1 \\
            3 & 0 & 0 & 1 & 1 & 1 \\
            4 & 0 & 1 & 0 & 0 & 1 \\
            5 & 0 & 1 & 0 & 1 & 1 \\
            6 & 0 & 1 & 1 & 0 & 1 \\
            7 & 0 & 1 & 1 & 1 & 1 \\
        \end{tabular}
        \quad
        \begin{tabular}{ccccc|c}
            & A & B & C & D & $H_s$ \\
            \cmidrule{2-6}
            8 & 1 & 0 & 0 & 0 & 0 \\
            9 & 1 & 0 & 0 & 1 & 0 \\
            10 & 1 & 0 & 1 & 0 & 1 \\
            11 & 1 & 0 & 1 & 1 & 1 \\
            12 & 1 & 1 & 0 & 0 & 1 \\
            13 & 1 & 1 & 0 & 1 & 1 \\
            14 & 1 & 1 & 1 & 0 & 1 \\
            15 & 1 & 1 & 1 & 1 & 1 \\
        \end{tabular}
    \end{center}
    \par With these two tables, it is clear that these two functions are
    equivalent. The truth tables are the same as well as the waveforms,
    signifying that the two functions provide the same functionality. The
    ability to do this simplification is required greatly since the less gates
    and the less inputs the gates are required to handle, the less power will be
    used and therefore the circuit and the chips in which these are implemented
    will function much more effectively and efficiently.
\end{document}
